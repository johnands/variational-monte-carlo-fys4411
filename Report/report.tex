\documentclass[english, a4paper]{article}

\usepackage[T1]{fontenc}    % Riktig fontencoding
\usepackage[utf8]{inputenc} % Riktig tegnsett
\usepackage{babel}   % Ordelingsregler, osv
\usepackage{graphicx}       % Inkludere bilder
\usepackage{booktabs}       % Ordentlige tabeller
\usepackage{url}            % Skrive url-er
\usepackage{textcomp}       % Den greske bokstaven micro i text-mode
\usepackage{units}          % Skrive enheter riktig
\usepackage{float}          % Figurer dukker opp der du ber om
\usepackage{lipsum}         % Blindtekst
\usepackage{subcaption} 
\usepackage{amssymb}
\usepackage{color}
\usepackage{amsmath}  
\usepackage{braket} 
\usepackage{multicol}
%\usepackage[]{mcode}
\usepackage{listings}    %Add source code
\usepackage{amsfonts}
\usepackage{setspace}
\usepackage[cm]{fullpage}		% Smalere marger.
\usepackage{verbatim} % kommentarfelt.
\setlength{\columnseprule}{1pt}	%(width of separationline)
\setlength{\columnsep}{1.0cm}	%(space from separation line)
\newcommand\lr[1]{\left(#1\right)} 
\newcommand\lrb[1]{\left[#1\right]} 
\newcommand\bk[1]{\langle#1\rangle} 
\newcommand\uu[1]{\underline{\underline{#1}}} % Understreker dobbelt.



% JF i margen
\makeatletter
\makeatother
\newcommand{\jf}[1]{\subsubsection*{JF #1}\vspace*{-2\baselineskip}}

% Skru av seksjonsnummerering (-1)
\setcounter{secnumdepth}{3}

\begin{document}
\renewcommand{\figurename}{Figure}
% Forside
\begin{titlepage}
\begin{center}

\textsc{\Large FYS4411}\\[0.5cm]
\textsc{\Large Spring 2016}\\[1.5cm]
\rule{\linewidth}{0.5mm} \\[0.4cm]
{ \huge \bfseries Variational Monte Carlo studies of bosonic systems}\\[0.10cm]
\rule{\linewidth}{0.5mm} \\[1.5cm]

% Av hvem?
\begin{minipage}{0.49\textwidth}
    \begin{center} \large
        John-Anders Stende \\[0.8cm]
    \end{center}
\end{minipage}


\vfill

% Dato nederst
\large{Date: \today}

\end{center}
\end{titlepage}
%%%%%%%%%%%%%%%%%%%%%%%%%%%%%%%%%%%
%%%%%%%%%%%%%%%%%%%%%%%%%%%%%%%%%%%

%\begin{multicols*}{2}

\begin{abstract}
The aim of this project is to use the Variational Monte Carlo (VMC) method to evaluate the 
ground state energy of a trapped, hard sphere Bose gas for different numbers of particles
with a specific trial wave function. 

***Main findings***
\end{abstract}


\section*{Introduction}
Demonstrations of Bose-Einstein condensation (BEC) in gases of alkali atoms confined in magnetic traps has gained
a lot of interest in the scientific community in recent years. Of interest is for instance the fraction of condensed atoms, 
the nature of the condensate and the excitations above the condensate. \\

\noindent An important feature of the trapped alkali systems is that they are dilute, i.e. the effective atom size
is small compared to both the trap size and the inter-atomic spacing. In this situation the physics is
dominated by two-body collisions, well discribed in terms of the $s$-wave scattering length $a$ of the atoms.
The condition for diluteness is defined by the gas parameter $x(\mathbf{r}) = n(\mathbf{r})a^3$, where $n(\mathbf{r})$
is the local density of the system. The theoretical framework of the Gross-Pitaevski equation is valid for $x_{av} \leq 10^{-3}$,
but recent experiments have shown that the gas parameter may exceed this value due to the presence of so-called Feshbach resonance.
Therefore, other methods like the VMC method may be needed. \\

\noindent In this project we evaluate the ground state energy of a trapped BEC by simulating different numbers of bosons
in a harmonic oscillator potential in one, two and three dimensions. The energy is obtained using the VMC method, both with 
and without importance sampling. We have studied both the interacting and the non-interacting case, i.e. with both 
an uncorrelated and a correlated trial wave function. The method of blocking is utilized to do statistical analysis on
the numerical data. We optimize the variational parameter $\alpha$ using the steepest descent method. The one-body 
density in the interacting and non-interacting case is alsox computed. 


\section*{Theory}
The trap we use is a spherical (S) or an elliptical (E) harmonic trap in one, two and three dimensions, with the latter given by
\begin{equation}
 V_{ext}({\bf r}) = 
 \Bigg\{
\begin{array}{ll}
	 \frac{1}{2}m\omega_{ho}^2r^2 & (S)\\
 \strut
	 \frac{1}{2}m[\omega_{ho}^2(x^2+y^2) + \omega_z^2z^2] & (E)
 \label{trap_eqn}
\end{array}
\end{equation}
where $\omega_{ho}$ and $\omega_z$ defines the trap potential strength in the $xy$-plane and $z$-direction respectively.
The two-body Hamiltonian is
 \begin{equation}
     H = \sum_i^N \left(
	 \frac{-\hbar^2}{2m}
	 { \bigtriangledown }_{i}^2 +
	 V_{ext}({\bf{r}}_i)\right)  +
	 \sum_{i<j}^{N} V_{int}({\bf{r}}_i,{\bf{r}}_j),
 \end{equation}
 and we reresent the inter-boson interaction by a pairwise, repulsive potential
 \begin{equation}
 V_{int}(|{\bf r}_i-{\bf r}_j|) =  \Bigg\{
 \begin{array}{ll}
	 \infty & {|{\bf r}_i-{\bf r}_j|} \leq {a}\\
	 0 & {|{\bf r}_i-{\bf r}_j|} > {a}
 \end{array}
 \end{equation}
 where ${a}$ is the so-called hard-core diameter of the bosons.
 
 Our trial wave function for the ground state with N atoms is given by
  \begin{equation}
 \Psi_T({\bf R})=\Psi_T({\bf r}_1, {\bf r}_2, \dots {\bf r}_N,\alpha,\beta)=\prod_i g(\alpha,\beta,{\bf r}_i)\prod_{i<j}f(a,|{\bf r}_i-{\bf r}_j|),
 \label{eq:trialwf}
 \end{equation}
 where $\alpha$ and $\beta$ are variational parameters.
 The single-particle wave function is proportional to the harmonic
 oscillator function for the ground state, i.e.,
 \begin{equation}
    g(\alpha,\beta,{\bf r}_i)= \exp{[-\alpha(x_i^2+y_i^2+\beta z_i^2)]}.
 \end{equation}
 For spherical traps we have $\beta = 1$ and for non-interacting
 bosons ($a=0$) we have $\alpha = 1/2a_{ho}^2$.  The correlation wave
 function is
 \begin{equation}
    f(a,|{\bf r}_i-{\bf r}_j|)=\Bigg\{
 \begin{array}{ll}
	 0 & {|{\bf r}_i-{\bf r}_j|} \leq {a}\\
	 (1-\frac{a}{|{\bf r}_i-{\bf r}_j|}) & {|{\bf r}_i-{\bf r}_j|} > {a}.
 \end{array}
 \end{equation}
 
 
\subsection*{Analytical results}

The quantity we are aiming to compute is the expectiation value of the so-called local energy
 \begin{equation}
    E_L({\bf R})=\frac{1}{\Psi_T({\bf R})}H\Psi_T({\bf R}),
    \label{eq:locale}
 \end{equation}
We can find closed-form expressions for the local energy with our specific Hamiltonian $H$ and trial wavefunction $\Psi_T$.
Computing the local energy involves a second derivative of $\Psi_T$, which can be expensive to compute numerically. 
Analytical expressions are therefore useful, as they can speed up the computations.\\

\noindent First, we find the local energy with only the (spherical) harmonic oscillator potential, that is we set $a=0$ and $\beta=1$.
During these calcuations we will use natural units, thus $\hbar=m=1$.
For one particle in one dimension we have
\begin{equation}
\Psi_T(x) = e^{-\alpha x^2}
\end{equation}
and
\begin{equation}
    H =  
	 -\frac{1}{2}
	 \frac{\partial^2}{\partial x^2} +
	 \frac{1}{2}\omega x^2
\end{equation}
The second derivate of the trial wave function is
\begin{equation}
 \frac{\partial^2 \Psi_T}{\partial x^2} = 2\alpha e^{-\alpha x^2}(2\alpha x^2 - 1)
\end{equation}
so that
\begin{equation}
 E_L = \frac{1}{\Psi_T}H\Psi_T = \alpha(1 - 2\alpha x^2) + \frac{1}{2}\omega^2x^2
\end{equation}
In three dimensions the double derivative is replaced by the Laplacian when computing the kinetic energy
\begin{align}
 \bigtriangledown^2\Psi_T &= 2\alpha(2\alpha x^2 - 1) + 2\alpha(2\alpha y^2 - 1) + 2\alpha(2\alpha z^2 - 1) \\
                          &= 2\alpha(2\alpha r^2 - 3)
\end{align}
thus the local energy is
\begin{equation}
 E_L = -\frac{1}{2}\bigtriangledown^2\Psi_T + V_{ext} = \alpha(3 - 2\alpha r^2) + \frac{1}{2}\omega^2r^2 
\end{equation}
We now turn our attention to $N$ particles, with the following wavefunction and Hamiltonian
\begin{align}
 &\Psi_T({\bf R}) = \prod_i e^{-\alpha r_i^2} \\
 &H =     \sum_i^N \left(
	 -\frac{1}{2}
	 { \bigtriangledown }_{i}^2 +
	 \frac{1}{2}\omega^2r_i^2 \right)
\end{align}
The first term of the k-th Laplacian of this wavefunction is
\begin{equation}
 \bigtriangledown^2_k\prod_i e^{-\alpha r_i^2} = 2\alpha(2\alpha x_k^2 - 1)\prod_i e^{-\alpha r_i^2}
\end{equation}
and when we divide with $\Psi_T$ to obtain the the local energy we end up with
\begin{equation}
 E_L = \sum_i^N \left( \alpha(3 - 2\alpha r_i^2) + \frac{1}{2}\omega^2r_i^2 \right)
\end{equation}
For one dimension the expression is
\begin{equation}
 E_L = \sum_i^N \left( \alpha(1 - 2\alpha x_i^2) + \frac{1}{2}\omega^2x_i^2 \right)
\end{equation}
\\

\noindent It is also useful to compute the analytical expression for the drift force $F$ to be used in importance sampling
\begin{equation}
 F = \frac{2\nabla \Psi_T}{\Psi_T}.
\end{equation}
The gradient of $\Psi_T$ is
\begin{align}
 \nabla \Psi_T &= (-2\alpha x, -2\alpha y, -2\alpha z)e^{-\alpha r^2} \\
               &= -2\alpha e^{-\alpha r^2} \bf{r}
\end{align}
Dividing by the wavefunction yields
\begin{equation}
 F = -4\alpha \bf{r}
\end{equation}


Find local energy for the whole system...


\section*{Methods}

We use the \textit{Variational Monte Carlo} (VMC) method in this project to obtain the ground state energy
for our bosonic system. VMC applies the \textit{variational principle} from quantum mechanics
\begin{equation}
 E_0 \leq \frac{\langle \Psi_T | H | \Psi_T \rangle}{\langle \Psi_T | \Psi_T \rangle}
\end{equation}
which states that the ground state energy is always less or equal than the expectation value of our Hamiltonian $H$
for any trial wavefunction $\Psi_T$. VMC consists in choosing a trial wavefunction depending on one or more
variational parameters, and finding the values of these parameters for which the expectation value of the 
energy is the lowest possible. The main challenge is to compute the multidimensional integral
\begin{equation}
 \frac{\langle \Psi_T | H | \Psi_T \rangle}{\langle \Psi_T | \Psi_T \rangle} = \frac{\int d {\bf R} \Psi_T^* H \Psi_T}{\int d {\bf R} \Psi_T^* \Psi_T}
\end{equation}
Traditional integration methods like Gauss-Legendre methods are too computationally expensive, therefore 
other methods are needed.

\subsection*{Monte Carlo integration}

Monte Carlo integration employs a non-deterministic approach to evaluate multidimensional integrals like
\begin{equation}
 I = \int_\Omega f({\bf x}) d{\bf x}
\end{equation}
Instead of using an explicit integration scheme, we sample points
\begin{equation}
 {\bf x}_1 \dots {\bf x}_N \in \Omega
\end{equation}
according to some rule. The naive approach, called brute force Monte Carlo, is to use $N$ uniform samples. 
The integral can then be approximated as the average of the function values at these points
\begin{equation}
 I \approx \frac{1}{N} \sum_{i=1}^N f({\bf x}_i)
\end{equation}
The brute force method is however not very efficient, as it samples an equal amount of points in all regions of $\Omega$, 
including those where $f$ is zero. A more clever approach is to sample points according to the probability distribution (PDF)
defined by $f$, called \textit{importance sampling} (descibed in detail below). 
Such a PDF is in general difficult to obtain, but using a PDF that fits $f$ well can greatly increase accuracy and efficiency. 

























 

























 %\end{multicols*}
%%%%%%%%%%%%%%%%%%%%%%%%%%%%%%%%%%%
%%%%%%%%%%%%%%%%%%%%%%%%%%%%%%%%%%%
\end{document}

\begin{comment}

% deloppgave
\begin{enumerate}
\item[\bf a)]
\item[\bf b)]
\item[\bf c)]
\item[\bf d)]
\end{enumerate}

%%%%%%%%
% Tabell
\begin{table}[H]
  \centering
  \begin{tabular}{ | c | r | r | r | r | r |}
    \hline
    & & & & & \\*
    \hline
    & & & & & \\*
    \hline
  \end{tabular}
  \caption{some caption}
  \label{tab:Tabell1}
\end{table}

%%%%%%%%
% Enkel figur
\begin{figure}[H]
\begin{center}
  \includegraphics[width = 120mm]{/users/filiphl/Desktop/Studie/Emne/ObligX/filnavn.png}
  \caption{some caption}\label{fig:fig1}
  \end{center}
\end{figure}

%%%%%%%%
% 2 figurer sbs
\begin{figure}
\begin{minipage}[t]{0.48\linewidth}
  \includegraphics[width=\textwidth]{fil}
  \caption{}
  \label{fig:minipage1}
\end{minipage}
\quad
\begin{minipage}[t]{0.48\linewidth}
\includegraphics[width=\textwidth]{fil}
  \caption{}
  \label{fig:minipage1}
\end{minipage}
\end{figure}

%%%%%%%%
% X antall kollonner
\begin{multicols*}{X}
\begin{spacing}{0.7} % verticale mellomrom
%kan f.eks benytte align?
\end{spacing}
\end{multicols*}


%%%%%%%%
%Matrise
\begin{equation*}
    {\bf A} = \left(\begin{array}{cccccc}
                           z &z &z &z &z &z \\
                           z &z &z &z &z &z \\
                           z &z &z &z &z &z \\
                           z &z &z &z &z &z \\
                           z &z &z &z &z &z \\
                           z &z &z &z &z &z \\
                      \end{array} \right)
\end{equation*}
%%%%%%%%

